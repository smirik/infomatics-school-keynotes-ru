\documentclass[compress,red]{beamer}
\usepackage[utf8]{inputenc}
\usepackage{ucs}
\usepackage{amsmath}
\usepackage{amsfonts}
\usepackage{amssymb}
\usepackage[russian]{babel}
\usepackage{graphicx}
\usepackage{wrapfig}

\usepackage{tikz}
\usepackage{verbatim}

\usepackage{color}
\usepackage{xcolor}
\usepackage{listings}

\usepackage{caption}

\lstset{
language=ruby,
extendedchars=\true,
inputencoding=utf8x,
commentstyle=\itshape,
stringstyle=\bf,
belowcaptionskip=5pt }


\DeclareCaptionFont{white}{\color{white}}
\DeclareCaptionFormat{listing}{\colorbox{gray}{\parbox{\textwidth}{#1#2#3}}}
\captionsetup[lstlisting]{format=listing,labelfont=white,textfont=white}

\usetikzlibrary{calc,trees,positioning,arrows,chains,shapes.geometric,%
    decorations.pathreplacing,decorations.pathmorphing,shapes,%
    matrix,shapes.symbols}

\tikzset{
>=stealth',
  punktchain/.style={
    rectangle, 
    rounded corners, 
    % fill=black!10,
    draw=black, very thick,
    text width=10em, 
    minimum height=3em, 
    text centered, 
    on chain},
  line/.style={draw, thick, <-},
  element/.style={
    tape,
    top color=white,
    bottom color=blue!50!black!60!,
    minimum width=8em,
    draw=blue!40!black!90, very thick,
    text width=10em, 
    minimum height=1.5em, 
    text centered, 
    on chain},
  every join/.style={->, thick,shorten <=1pt},
  decoration={brace},
  tuborg/.style={decorate},
  tubnode/.style={midway, right=2pt},
}

\mode<presentation>

\usetheme{Warsaw}

\definecolor{Red}{rgb}{1,0,0}
\definecolor{Blue}{rgb}{0,0,1}
\definecolor{Green}{rgb}{0,1,0}
\definecolor{magenta}{rgb}{1,0,.6}
\definecolor{lightblue}{rgb}{0,.5,1}
\definecolor{lightpurple}{rgb}{.6,.4,1}
\definecolor{gold}{rgb}{.6,.5,0}
\definecolor{orange}{rgb}{1,0.4,0}
\definecolor{hotpink}{rgb}{1,0,0.5}
\definecolor{newcolor2}{rgb}{.5,.3,.5}
\definecolor{newcolor}{rgb}{0,.3,1}
\definecolor{newcolor3}{rgb}{1,0,.35}
\definecolor{darkgreen1}{rgb}{0, .35, 0}
\definecolor{darkgreen}{rgb}{0, .6, 0}
\definecolor{darkred}{rgb}{.75,0,0}

\xdefinecolor{olive}{cmyk}{0.64,0,0.95,0.4}
\xdefinecolor{purpleish}{cmyk}{0.75,0.75,0,0}

\useoutertheme[subsection=false]{smoothbars}

\title{Задачи по ООП}
\author{Информатика \\ 10-11 классы}

%\usecolortheme{dolphin}


\begin{document}
%%титульная страница
\maketitle
%% основные моменты

\section{Задание 1}

\subsection{Задание}
\begin{frame}[fragile]
  \frametitle{Задание}
  \begin{itemize}
    \item Создать следующие классы: человек, ученик, ученик--раздолбай, учитель, директор. 
    \item Каждый человек имеет: фамилию, имя, отчество, год рождения. Наследование определено в соответствии со здравым смыслом (ученик--раздолбай --- наследник ученика). Все сущности имеют методы:
     \begin{enumerate}
      \item Посчитать возраст (getAges).
      \item обратиться по имени (getName) по правилу: учитель и директор --- имя + отчество, ученик --- имя, ученик-раздолбай --- ``Бяка'' + имя. 
      \item булевский метод главный (head?): для директора возвращается истина, для остальных --- ложь.
     \end{enumerate} 
    \item ФИО и год рождения должно задаваться в конструкторе.
    \item После реализации создать экземпляры каждого класса и вызвать для них методы getName, getAges, head?.
  \end{itemize}
\end{frame}

\subsection{Задание 1-1}
\begin{frame}[fragile]
  \frametitle{Шаг 1}
  \begin{itemize}
    \item Прежде всего, создадим класс Person.
    \item В классе есть четыре свойства: first\_name, last\_name, middle\_name, birthday
  \end{itemize}
  
  \scriptsize{
  \begin{lstlisting}[language=ruby,basicstyle=\footnotesize,label=ruby1,caption=Основа Person]
    class Person
      attr_accessor :first_name, :last_name,
                    :middle_name, :birthday
    end
  \end{lstlisting}
  }
\end{frame}

\subsection{Задание 1-2}
\begin{frame}[fragile]
  \frametitle{Шаг 2}
  \begin{itemize}
    \item В условии задачи требуется, чтобы основные свойства класса задавались сразу.
    \item То есть, мы хотим, чтобы работал следующий код:
  \end{itemize}
  \scriptsize{
  \begin{lstlisting}[language=ruby,basicstyle=\footnotesize,label=ruby2,caption=Пример объекта]
    p = Person.new("Иванов", "Иван", "Иванович", 1975)
  \end{lstlisting}
  }
\end{frame}

\subsection{Задание 1-3}
\begin{frame}[fragile]
  \frametitle{Шаг 3}
  \begin{itemize}
    \item Для того, чтобы при создании можно было сразу задать какие-либо параметры, нам нужно определить конструктор.
    \item Конструктор на вход (в соответствии с кодом предыдущего слайда) будет принимать 4 аргумента.
  \end{itemize}
  \scriptsize{
  \begin{lstlisting}[language=ruby,basicstyle=\footnotesize,label=ruby3,caption=Конструктор]
  class Person
    attr_accessor :first_name, :last_name,
                  :middle_name, :birthday
    def initialize(fname, lname, mname, birthday)
      @first_name  = fname
      @last_name   = lname
      @middle_name = mname
      @birthday    = birthday 
    end
  end    
  \end{lstlisting}
  }
\end{frame}

\subsection{Задание 1-4}
\begin{frame}[fragile]
  \frametitle{Шаг 4}
  \begin{itemize}
    \item Возраст у экземпляров класса Person и у его наследников будет считаться всегда одинаково.
    \item Поэтому определим соответствующий метод в самом классе Person.
    \item Через механизм наследования метод автоматически будет доступен всем наследникам.
  \end{itemize}
  \scriptsize{
  \begin{lstlisting}[language=ruby,basicstyle=\footnotesize,label=ruby4,caption=Метод age]
    class Person
      ...
      def age
        2012 - @birthday
      end
    end
  \end{lstlisting}
  }
\end{frame}

\subsection{Задание 1-5}
\begin{frame}[fragile]
  \frametitle{Шаг 5}
  \begin{itemize}
    \item В большинстве случаев метод head? будет возвращать ложь (за исключением объекта класса Директор).
    \item Поэтому создадим базовый метод в классе Person, а в классе Headmaster используем полиморфизм (переопределение метода) для изменения результата.
  \end{itemize}
  \scriptsize{
  \begin{lstlisting}[language=ruby,basicstyle=\footnotesize,label=ruby5,caption=Метод head]
    class Person
      ...
      def head?
        false
      end
    end
  \end{lstlisting}
  }
\end{frame}

\subsection{Задание 1-6}
\begin{frame}[fragile]
  \frametitle{Шаг 6}
  \begin{itemize}
    \item Стандартное обращение к человеку --- по имени--отчеству. 
    \item Раз стандартное --- значит, определяем в классе--родителе. При необходимости используем полиморфизм.
  \end{itemize}
  
  \scriptsize{
  \begin{lstlisting}[language=ruby,basicstyle=\footnotesize,label=ruby6,caption=Метод name]
    class Person
      ...
      def name
        @first_name + " " + @middle_name
      end
    end
  \end{lstlisting}
  }
\end{frame}

\subsection{Задание 1-7}
\begin{frame}[fragile]
  \frametitle{Шаг 7}
  \begin{itemize}
    \item Проверим класс Person, вызвав последовательно все методы.
  \end{itemize}
  \scriptsize{
  \begin{lstlisting}[language=ruby,basicstyle=\footnotesize,label=ruby7,caption=Person]
    p = Person.new("Иванов", "Иван", "Иванович", 1975)
    puts p.name
    puts p.age
    puts p.head?
  \end{lstlisting}
  }
\end{frame}

\subsection{Задание 1-8}
\begin{frame}[fragile]
  \frametitle{Шаг 8}
  \begin{itemize}
    \item Класс Teacher имеет абсолютно стандартную реализацию.
    \item Все методы в нём совпадают с методами Person.
    \item Поэтому достаточно просто его определить.
  \end{itemize}
  
  \scriptsize{
  \begin{lstlisting}[language=ruby,basicstyle=\footnotesize,label=ruby8,caption=Teacher]
    class Teacher < Person
    end
  \end{lstlisting}
  }
  
\end{frame}

\subsection{Задание 1-9}
\begin{frame}[fragile]
  \frametitle{Шаг 9}
  \begin{itemize}
    \item У ученика другое обращение.
    \item Используем полиморфизм для переопределения метода \textbf{name}.
  \end{itemize}
  \scriptsize{
  \begin{lstlisting}[language=ruby,basicstyle=\footnotesize,label=ruby9,caption=Student]
    class Student < Person
      def name
        @first_name
      end
    end
  \end{lstlisting}
  }
\end{frame}

\subsection{Задание 1-10}
\begin{frame}[fragile]
  \frametitle{Шаг 10}
  \begin{itemize}
    \item Класс BadStudent также имеет отличное ото всех обращение с приставкой <<Бяка>>.
  \end{itemize}
  \scriptsize{
  \begin{lstlisting}[language=ruby,basicstyle=\footnotesize,label=ruby10,caption=BadStudent]
    class BadStudent < Student
      def name
        "Byaka " + @first_name
      end
    end
  \end{lstlisting}
  }
\end{frame}

\subsection{Задание 1-11}
\begin{frame}[fragile]
  \frametitle{Шаг 11}
  \begin{itemize}
    \item Класс Headmaster имеет стандартное обращение по имени--отчеству.
    \item Значит, метод name переопределять не надо.
    \item А вот метод head? должен возвращать истину.
  \end{itemize}
  \scriptsize{
  \begin{lstlisting}[language=ruby,basicstyle=\footnotesize,label=ruby11,caption=Headmaster]
    class Headmaster < Person
      def head?
        true
      end
    end
  \end{lstlisting}
  }
  
\end{frame}

\section{Задание 2}

\subsection{Сложное задание}
\begin{frame}[fragile]
  \frametitle{Сложное задание}
  \begin{itemize}
    \item Реализовать класс \textbf{Двумерный Вектор}. 
    \item Класс имеет два свойства: x-компонента, y-компонента. 
    \item Методы класса:
      \begin{enumerate}
        \item посчитать длину (модуль)
        \item прибавить к текущему вектору другой
        \item отнять от текущего вектора другой
        \item изменить знак вектора (-вектор)
        \item умножить вектор на скаляр (вещественное число)
        \item скалярно умножить на другой вектор
      \end{enumerate}
  \end{itemize}
\end{frame}

\subsection{Задание 2-1}
\begin{frame}[fragile]
  \frametitle{Шаг 1}
  \begin{itemize}
    \item Прежде всего, определимся, что есть у вектора.
    \item Двумерный вектор --- это два числа (x,y) (аналог радиус--вектора в геометрии).
    \item Других свойств у вектора нет. Но как хранить эти?
    \item Два варианта:
      \begin{enumerate}
        \item В свойствах :x, :y.
        \item В едином свойстве :coords
      \end{enumerate}
    \item Второй вариант универсальнее, поэтому будем использовать его.
    \item \textbf{NB}. Кстати, мы сделаем программу для векторов любой размерности.
  \end{itemize}
\end{frame}

\subsection{Задание 2-2}
\begin{frame}[fragile]
  \frametitle{Шаг 2}
  \begin{itemize}
    \item В конструкторе \textbf{по умолчанию} зададим значение координат в виде пустого массива.
  \end{itemize}
  \scriptsize{
  \begin{lstlisting}[language=ruby,basicstyle=\footnotesize,label=ruby12,caption=Vector]
    class Vector
      attr_accessor :coords
      
      def initialize(coords = [])
        @coords = coords
      end
    end
    
    v = Vector.new([1, 2])
  \end{lstlisting}
  }
\end{frame}

\subsection{Задание 2-3}
\begin{frame}[fragile]
  \frametitle{Шаг 3}
  \begin{itemize}
    \item Определим метод подсчёта модуля.
    \item Модуль вычисляется по теореме Пифагора:
    $$
      \sqrt{\sum_i coords[i]^2}
    $$
  \end{itemize}
  \scriptsize{
  \begin{lstlisting}[language=ruby,basicstyle=\footnotesize,label=ruby13,caption=Вычисление модуля]
    class Vector
      ..
      def module
        (@coords.inject(0){|res, elem| res+elem**2})**0.5
      end
    end
  \end{lstlisting}
  }
\end{frame}

\subsection{Задание 2-4}
\begin{frame}[fragile]
  \frametitle{Шаг 4}
  \begin{itemize}
    \item Заметим, что модуль будет вычисляться не только для двумерного вектора, но и для вектора любой размерности.
    \item Теперь определим операцию сложения.
    \item При сложении компоненты векторов также складываются.
    \item \textbf{Важно}: операция сложения должна возвращать новый вектор --- сумму текущего и того, с которым складываем.
    \item Сложение делаем покоординатно.
  \end{itemize}
\end{frame}

\subsection{Задание 2-4-1}
\begin{frame}[fragile]
  \frametitle{Шаг 4, часть 2}
  \scriptsize{
  \begin{lstlisting}[language=ruby,basicstyle=\footnotesize,label=ruby14,caption=Сложение]
  def +(v)
    sum = Vector.new
    size = @coords.size-1
    for i in 0..size
      sum.coords[i] = @coords[i] + v.coords[i]
    end
    sum
  end
  \end{lstlisting}
  }
\end{frame}

\subsection{Задание 2-4-2}
\begin{frame}[fragile]
  \frametitle{Шаг 4, часть 2, альтернатива}
  \begin{itemize}
    \item Для упрощения записи используем метод each\_index, который проходит по каждому индексу массива.
  \end{itemize}
  \scriptsize{
  \begin{lstlisting}[language=ruby,basicstyle=\footnotesize,label=ruby15,caption=Упрощение]
  def +(v)
    sum = Vector.new
    @coords.each_index{|i| sum.coords[i] = 
      @coords[i]+v.coords[i]}
    sum
  end    
  \end{lstlisting}
  }
  
\end{frame}

\subsection{Задание 2-5}
\begin{frame}[fragile]
  \frametitle{Шаг 5}
  \begin{itemize}
    \item Операция умножения немного сложнее.
    \item Если мы умножаем на число, то надо все компоненты вектора умножить на данное число.
    \item Если же мы умножаем вектор на вектор, то надо возвращать скалярное произведение.
    \item Как отличить, что нам передаётся в качестве аргумента?
    \item Используем метод \textbf{class}, который возвращает строку с названием класса.
  \end{itemize}
\end{frame}

\subsection{Задание 2-5-1}
\begin{frame}[fragile]
  \frametitle{Шаг 5, часть 2}
  \scriptsize{
  \begin{lstlisting}[language=ruby,basicstyle=\footnotesize,label=ruby16,caption=Умножение]
    def *(v)
      if (v.class == Vector)
        product = 0
        @coords.each_index{|i| product+=
          @coords[i]*v.coords[i]}
      else
        product = Vector.new
        @coords.each_index{|i| product.coords[i] =
          @coords[i]*v}
      end
      product
    end
  \end{lstlisting}
  }
  
\end{frame}

\subsection{Задание 2-6}
\begin{frame}[fragile]
  \frametitle{Шаг 6}
  \begin{itemize}
    \item Аналогично делаем остальные методы. 
    \item Например, метод <<отрицание>>.
  \end{itemize}
  
  \scriptsize{
  \begin{lstlisting}[language=ruby,basicstyle=\footnotesize,label=ruby17,caption=Отрицание]
    def -@
      self.coords = self.coords.map{|i| -i}
    end
  \end{lstlisting}
  }
  
\end{frame}

\subsection{Задание 2-7}
\begin{frame}[fragile]
  \frametitle{Шаг 7}
  \scriptsize{
  \begin{lstlisting}[language=ruby,basicstyle=\footnotesize,label=ruby18,caption=Упрощение]
  def -(v)
    sum = Vector.new
    @coords.each_index{|i| sum.coords[i] = 
      @coords[i]-v.coords[i]}
    sum
  end    
  \end{lstlisting}
  }  
\end{frame}

\subsection{Мысль}
\begin{frame}
  \begin{center}
    \Huge{Мысль}
  \end{center}
  \begin{center}
    \Large{Класс Vector очень и очень похож на класс Array}
  \end{center}
\end{frame}

\subsection{Расшифровка}
\begin{frame}[fragile]
  \frametitle{Расшифровка мысли}
  \begin{itemize}
    \item Класс Vector, как мы определили на шаге 1, имеет только одно свойство --- координаты.
    \item Координаты представляют собой массив.
    \item Раз всё в ruby --- объекты, значит, и массивы тоже.
    \item А, значит, логично было бы вместо определения класса Vector, отнаследовать его от массива.
  \end{itemize}
    \scriptsize{
    \begin{lstlisting}[language=ruby,basicstyle=\footnotesize,label=ruby19,caption=Вектор vs Массив]
      class Vector < Array
      end
    \end{lstlisting}
    }
\end{frame}

\subsection{С массивом}
\begin{frame}[fragile]
  \frametitle{Работа с массивом}
  \begin{itemize}
    \item Теперь мы можем обращаться к i-ой координате внутри класса так: \textbf{self[i]}.
    \item Пример метода:
  \end{itemize}
    \scriptsize{
    \begin{lstlisting}[language=ruby,basicstyle=\footnotesize,label=ruby20, caption=Вектор как массив]
      class Vector < Array
        def +(a)
          sum = Vector.new
          self.each_index{|k| sum[k] = self[k]+a[k]}
          sum
        end      
      end
    \end{lstlisting}
    }
\end{frame}

\subsection{Создание вектора}
\begin{frame}[fragile]
  \frametitle{Создание векторомассива}
  \scriptsize{
  \begin{lstlisting}[language=ruby,basicstyle=\footnotesize,label=ruby21,caption=Создание]
    v1 = Vector.new[1,2,3]
    v2 = Vector.new[3,4,5]
    v3 = v1+v3
    puts v3.inspect
  \end{lstlisting}
  }
  
\end{frame}

\section{References}
\subsection{References}
\begin{frame}[fragile]
  \frametitle{References}
  \begin{itemize}
    \item При подготовке данного материала использовались сайты: http://ru.wikibooks.org/wiki/Ruby, http://rubydev.ru, http://en.wikipedia.org, http://ruby-lang.org.
    \item Все презентации доступны на http://school.smirik.ru!
    \item Вопросы, предложения, д/з: smirik@gmail.com
  \end{itemize}
\end{frame}

\end{document}