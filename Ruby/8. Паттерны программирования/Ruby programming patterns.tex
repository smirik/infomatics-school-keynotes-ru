\documentclass[compress,red]{beamer}
\usepackage[utf8]{inputenc}
\usepackage{ucs}
\usepackage{amsmath}
\usepackage{amsfonts}
\usepackage{amssymb}
\usepackage[russian]{babel}
\usepackage{graphicx}
\usepackage{wrapfig}

\usepackage{tikz}
\usepackage{verbatim}

\usepackage{color}
\usepackage{xcolor}
\usepackage{listings}

\usepackage{caption}

\lstset{
language=ruby,
extendedchars=\true,
inputencoding=utf8x,
commentstyle=\itshape,
stringstyle=\bf,
belowcaptionskip=5pt }


\DeclareCaptionFont{white}{\color{white}}
\DeclareCaptionFormat{listing}{\colorbox{gray}{\parbox{\textwidth}{#1#2#3}}}
\captionsetup[lstlisting]{format=listing,labelfont=white,textfont=white}

\usetikzlibrary{calc,trees,positioning,arrows,chains,shapes.geometric,%
    decorations.pathreplacing,decorations.pathmorphing,shapes,%
    matrix,shapes.symbols}

\tikzset{
>=stealth',
  punktchain/.style={
    rectangle, 
    rounded corners, 
    % fill=black!10,
    draw=black, very thick,
    text width=10em, 
    minimum height=3em, 
    text centered, 
    on chain},
  line/.style={draw, thick, <-},
  element/.style={
    tape,
    top color=white,
    bottom color=blue!50!black!60!,
    minimum width=8em,
    draw=blue!40!black!90, very thick,
    text width=10em, 
    minimum height=1.5em, 
    text centered, 
    on chain},
  every join/.style={->, thick,shorten <=1pt},
  decoration={brace},
  tuborg/.style={decorate},
  tubnode/.style={midway, right=2pt},
}

\mode<presentation>

\usetheme{Warsaw}

\definecolor{Red}{rgb}{1,0,0}
\definecolor{Blue}{rgb}{0,0,1}
\definecolor{Green}{rgb}{0,1,0}
\definecolor{magenta}{rgb}{1,0,.6}
\definecolor{lightblue}{rgb}{0,.5,1}
\definecolor{lightpurple}{rgb}{.6,.4,1}
\definecolor{gold}{rgb}{.6,.5,0}
\definecolor{orange}{rgb}{1,0.4,0}
\definecolor{hotpink}{rgb}{1,0,0.5}
\definecolor{newcolor2}{rgb}{.5,.3,.5}
\definecolor{newcolor}{rgb}{0,.3,1}
\definecolor{newcolor3}{rgb}{1,0,.35}
\definecolor{darkgreen1}{rgb}{0, .35, 0}
\definecolor{darkgreen}{rgb}{0, .6, 0}
\definecolor{darkred}{rgb}{.75,0,0}

\xdefinecolor{olive}{cmyk}{0.64,0,0.95,0.4}
\xdefinecolor{purpleish}{cmyk}{0.75,0.75,0,0}

\useoutertheme[subsection=false]{smoothbars}

\title{Паттерны программирования}
\author{Информатика \\ 10-11 классы}

%\usecolortheme{dolphin}


\begin{document}
%%титульная страница
\maketitle
%% основные моменты

\section{Введение}
\subsection{Введение}
\begin{frame}[fragile]
\frametitle{Паттерны программирования}
		\begin{itemize}
		\item Одним из отличительных свойств хорошего программиста является умение решать стандартные задачи стандартными методами.
		\item Практически любая задача в своём итоге сводится к нескольким проблемам:
		  \begin{enumerate}
				\item Считать исходные данные.
				\item Придумать способ хранения данных внутри программы.
				\item Понять, как проще всего разбить на подзадачи.
				\item Проверить, нет ли готовых решений?
				\item Преобразовать получившийся результат к требуемому.
		  \end{enumerate}
		\end{itemize}
\end{frame}

\section{Простые паттерны}
\subsection{Паттерн 1}
\begin{frame}
	\begin{center}
		\Huge{Паттерн 1}
	\end{center}
	\begin{center}
		\Large{Запомнить нужное значение}
	\end{center}
\end{frame}

\subsection{Паттерн 1, условие}
\begin{frame}[fragile]
\frametitle{Запомнить нужное значение}
		\begin{itemize}
			\item Часто в задаче требуется найти какую-то величину.
			\item К примеру, задача: найти наибольший элемент массива.
			\item Конечно, можно и нужно воспользоваться методом \emph{max}, однако попробуем решить эту задачу без читов.
			\item В данном случае нам нужно найти максимальный элемент.
			\item Если что-то надо найти, нужна переменная, куда мы будем записывать значение найденного.
			\item В данной задаче назовём её \emph{max}.
		\end{itemize}
\end{frame}

\subsection{Паттерн 1, часть 2}
\begin{frame}[fragile]
\frametitle{Алгоритм для паттерна 1}
	\begin{enumerate}
		\item Пройтись по всему массиву и рассмотрим последовательно каждый его элемент.
		\item В переменную \emph{max} будем записывать максимальный элемент \textbf{на текущий момент}.
		\item Если вдруг очередной элемент массива больше текущего максимального \emph{max}, запишем его значение в переменную \emph{max}.
		\item Единственно, возникает вопрос: а какое число изначально записать в \emph{max}?
		\item Если мы сделаем $max = 0$, то программа будет неверно работать для массивов, состоящих из отрицательных чисел.
	\end{enumerate}
\end{frame}

\subsection{Правило 1}
\begin{frame}
	\begin{center}
		\Huge{Правило 1}
	\end{center}
	\begin{center}
		\Large{Думайте, всегда ли программа будет работать правильно. Из-за ``забытых'' случаев падают космические корабли. }
	\end{center}
\end{frame}

\subsection{Паттерн 1, часть 3}
\begin{frame}[fragile]
\frametitle{Алгоритм для паттерна 1}
	\begin{itemize}
		\item \textbf{Решение}: очевидно, что максимальный элемент массива больше либо равен нулевого элемента (так как он --- максимальный).
		\item Поэтому если массив не пуст, то в качестве \emph{значения по умолчанию} можно взять именно его.
	\end{itemize}
		\scriptsize{
	  \begin{lstlisting}[label=ruby1,language=Ruby,caption=Паттерн 1]
			def max(array)
				 	return false if array.empty?
				 	max = array[0]
				 	for i in 0..array.size-1
				 	 	max = array[i] if (array[i] > max)
				 	end
				 	max
			end
	  \end{lstlisting}
		}
\end{frame}

\subsection{Паттерн 1, часть 4}
\begin{frame}[fragile]
\frametitle{Улучшения паттерна 1}
	\begin{itemize}
		\item Заметим, что при каждой итерации цикла нам приходится вычислять значение длины массива \emph{array.size}.
		\item Конечно, современные языки умеют \emph{кэшировать} такие операции, но лучше не полагаться на это.
		\item \textbf{Задача}. Как сделать так, чтобы не вычислять длину массива на каждом шаге?
		\item Вычислить её единожды! А результат записать в дополнительную переменную!
		\item Да, кстати: почему \emph{array.size-1}. Откуда взялась --1?
		\item Массивы в ruby нумеруются с нуля. Поэтому для массива, состоящего из $n$ элементов, ключ последнего будет равен $n-1$.
	\end{itemize}
\end{frame}

\subsection{Паттерн 1, часть 5}
\begin{frame}[fragile]
\frametitle{Улучшения для паттерна 1}
		\scriptsize{
	  \begin{lstlisting}[label=ruby2,language=Ruby,caption=Паттерн 1]
			def max(array)
				 	return false if array.empty?
				 	max = array[0]
				 	size = array.size-1
				 	for i in 0..size
				 	 	max = array[i] if (array[i] > max)
				 	end
				 	max
			end
	  \end{lstlisting}
		}
\end{frame}

\section{Паттерн 2}

\subsection{Паттерн 2}
\begin{frame}
	\begin{center}
		\Huge{Паттерн 2}
	\end{center}
	\begin{center}
		\Large{Ключ и значение максимального элемента.}
	\end{center}
\end{frame}

\subsection{Паттерн 2, условие}
\begin{frame}[fragile]
\frametitle{Ключ и значение максимального элемента}
		\begin{itemize}
			\item Усложним чуть-чуть задачу.
			\item \textbf{Задача}: найти ключ и значение наибольшего элемента массива.
			\item Здесь уже стандартный метод \emph{max} не поможет, так как он находит только значение, а не ключ.
			\item Конечно, есть и другие стандартные методы, но мы опять сделаем вручную.
			\item Нам нужно найти уже два числа: ключ и значение. 
			\item В данной задаче назовём их \emph{max\_key} и \emph{max\_value}.
		\end{itemize}
\end{frame}

\subsection{Паттерн 2, проблемы}
\begin{frame}[fragile]
\frametitle{Изменения по сравнению с паттерном 1}
		\begin{itemize}
			\item Первое изменение: если мы нашли элемент, который больше текущего максимального, перезаписать нужно не только значение \emph{array[i]}, но и соответствующий значению ключ \emph{i}.
			\item Второе изменение: функция должна возвращать не только максимальное значение, но и ключ.
			\item Самый простой способ вернуть несколько значений --- через массив.
			\item Можно, конечно, вернуть и единичный хэш ``ключ--значение'', в зависимости от общего стиля программы.
		\end{itemize}
\end{frame}

\subsection{Паттерн 2, часть 3}
\begin{frame}[fragile]
\frametitle{Программа для паттерна 2}
		\scriptsize{
	  \begin{lstlisting}[label=ruby3,language=Ruby,caption=Паттерн 2]
def max(array)
  return false if array.empty?
  max_key   = 0
  max_value = array[0]
  size = array.size-1
  for i in 0..size
    if (array[i] > max)
      max_key = i
      max_value = array[i]
    end
  end
  [max_key, max_value]
end
	  \end{lstlisting}
		}
\end{frame}

\section{Паттерн 3}

\subsection{Паттерн 3}
\begin{frame}
	\begin{center}
		\Huge{Паттерн 3}
	\end{center}
	\begin{center}
		\Large{Когда нужны булевские переменные.}
	\end{center}
\end{frame}

\subsection{Паттерн 3, условие}
\begin{frame}[fragile]
\frametitle{Поиск отрицательного элемента}
		\begin{itemize}
			\item Допустим, перед нами стоит задача узнать, есть ли в массиве отрицательный элемент.
			\item Конечно, мы бы могли воспользоваться ``магическими'' методами:
			\item Решим эту задачу без методов \emph{find\_all} и \emph{elem}.
			\item В нашей задаче ответ бинарный: есть или нет. Если в задаче или подзадаче нужен такой ответ, значит, нужна булевская переменная.
			\item Её можно назвать \emph{has\_negative} или по старой традиции \emph{flag} (аналог флажка, который либо опущен, лиоб поднят.
  		\scriptsize{
  	  \begin{lstlisting}[label=ruby4,language=Ruby,caption=Паттерн 3]
      puts "Есть" if array.find_all{|elem| elem<0}.any?
  	  \end{lstlisting}			
			}
		\end{itemize}
\end{frame}

\subsection{Паттерн 3, алгоритм}
\begin{frame}[fragile]
\frametitle{Алгоритм решения паттерна 3}
		\begin{enumerate}
			\item Пройдёмся по всему массиву. Изначально \emph{has\_negative} присвоим ложь, так как ни одного отрицательного числа мы пока не нашли.
			\item Проверим пробегаемый элемент, больше он или меньше нуля.
			\item Если он меньше нуля, то ``опустим флажок'', сделав \emph{has\_negative} равным истине и прервём цикл.
			\item Итого, если в конце цикла \emph{has\_negative} имеет значение ИСТИНА, то как минимум один отрицательный элемент найден. Иначе --- нет.
			\item Заметим, что переменная \emph{has\_negative} как раз и будет отвечать на вопрос, есть ли в массиве отрицательный элемент. Поэтому функцией можно просто возвращать её значение.
		\end{enumerate}
\end{frame}

\subsection{Паттерн 3, программа}
\begin{frame}[fragile]
\frametitle{Программа для паттерна 3}
		\scriptsize{
	  \begin{lstlisting}[label=ruby5,language=Ruby,caption=Паттерн 3]
def array_has_negative(array)
  return false if array.empty?
  has_negative = false
  size = array.size-1
  for i in 0..size
    if (array[i] < 0)
      has_negative = true
      break
    end
  end
  has_negative
end
	  \end{lstlisting}
		}
\end{frame}

\subsection{Паттерн 3, усложнение}
\begin{frame}[fragile]
\frametitle{Усложним задачу}
		\begin{itemize}
			\item Усложним задачу. Допустим, у нас есть массив чисел, содержащий элементы от 1 до 100. Сколько --- неизвестно. Повторы также возможны.
			\item Требуется вывести на экран те натуральные числа от 1 до 100, которые не встречаются в данном массиве.
			\item По сути, нам нужно ответить на 100 вопросов: есть ли в массиве число 1, есть ли в массиве число 2 и т.п.
			\item \textbf{Решение перебором в лоб:} пройтись циклом от 1 до 100 и проверить, есть ли в массиве пробегаемое число. 
			\item \textbf{Недостаток:} ооочень долгое время работы. Нам придётся совершить $n\cdot 100$ (по 100 проходов для каждого из n чисел) итераций.
			\item Можно улучшить метод, заведя булевский массив длиной 100. И, пробегая всего лишь один раз по всему массиву
		\end{itemize}
\end{frame}

\subsection{Паттерн 3, усложнение 2}
\begin{frame}[fragile]
\frametitle{Усложним задачу}
		\begin{itemize}
			\item Можно улучшить метод, заведя булевский массив длиной 100. Заполним его изначально ложью. 
			\item Пройдём по начальному массиву.
			\item Будем помечать истиной элементы с такими ключами, которые встречаются в виде значений в массиве.
			\item Дубликаты нам не страшны, так как двойное присваивание ничего не поменяет.
			\item Кстати, от них можно избавиться с помощью метода \emph{arr.uniq}.
			\item В конце просто пройдёмся по булевскому массиву и выведем на экран ключи тех элементов, которые равны лжи.
		\end{itemize}
\end{frame}

\subsection{Паттерн 3, программа}
\begin{frame}[fragile]
\frametitle{Программа для усложнённого паттерна 3}
		\scriptsize{
	  \begin{lstlisting}[label=ruby6,language=Ruby,caption=Усложнённый паттерн 3]
def missen_numbers(array)
  return (1..100).to_a if array.empty?
  size = array.size-1
  has_numbers = []
  100.times { |i| has_numbers[i] = false }
  for i in 0..size
    has_numbers[array[i]-1] = true
  end
  numbers = []
  100.times { |i| numbers[] = i+1 unless has_numbers[i] }
  numbers
end
	  \end{lstlisting}
		}
\end{frame}

\subsection{Правило 2}
\begin{frame}
	\begin{center}
		\Huge{Правило 2}
	\end{center}
	\begin{center}
		\Large{Никогда не забывайте, что элементы массива нумеруются с нуля. }
	\end{center}
\end{frame}

\subsection{Правило 3}
\begin{frame}
	\begin{center}
		\Huge{Правило 3}
	\end{center}
	\begin{center}
		\Large{Обязательно инициализируйте пустые массивы нужной длины.}
	\end{center}
\end{frame}

\subsection{Паттерн 3, усложнение пояснения}
\begin{frame}[fragile]
\frametitle{Необходимые пояснения}
		\begin{itemize}
			\item Почему мы присваиваем истине \emph{has\_numbers[array[i]-1]}. Откуда минус единица?
			\item Известно, что значения массива --- это числа от 1 до 100. У нас массив состоит из 100 элементов.
			\item То есть, ключи определены от 0 до 99. Вычитая единицу, мы ``переводим'' одно представление в другое.
			\item По аналогичным соображениям в последнем цикле \emph{times} мы прибавляем к ключу единицу (обратная операция).
			\item \textbf{По правилу 3}: когда вы пишите конструкцию вида \emph{arr[i] = arr[i] + 1}, предполагается, что $i$--ый элемент массива существует и определён. Если это не так, вы получите ошибку.
		\end{itemize}
\end{frame}

\section{References}
\subsection{References}
\begin{frame}[fragile]
  \frametitle{References}
  \begin{itemize}
    \item Все презентации доступны на http://school.smirik.ru!
    \item Вопросы, предложения, д/з: smirik@gmail.com
  \end{itemize}
\end{frame}

\end{document}