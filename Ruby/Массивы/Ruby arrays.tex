\documentclass[compress,red]{beamer}
\usepackage[utf8]{inputenc}
\usepackage{ucs}
\usepackage{amsmath}
\usepackage{amsfonts}
\usepackage{amssymb}
\usepackage[russian]{babel}
\usepackage{graphicx}
\usepackage{wrapfig}

\usepackage{tikz}
\usepackage{verbatim}

\usepackage{color}
\usepackage{xcolor}
\usepackage{listings}

\usepackage{caption}
\DeclareCaptionFont{white}{\color{white}}
\DeclareCaptionFormat{listing}{\colorbox{gray}{\parbox{\textwidth}{#1#2#3}}}
\captionsetup[lstlisting]{format=listing,labelfont=white,textfont=white}

\usetikzlibrary{calc,trees,positioning,arrows,chains,shapes.geometric,%
    decorations.pathreplacing,decorations.pathmorphing,shapes,%
    matrix,shapes.symbols}

\tikzset{
>=stealth',
  punktchain/.style={
    rectangle, 
    rounded corners, 
    % fill=black!10,
    draw=black, very thick,
    text width=10em, 
    minimum height=3em, 
    text centered, 
    on chain},
  line/.style={draw, thick, <-},
  element/.style={
    tape,
    top color=white,
    bottom color=blue!50!black!60!,
    minimum width=8em,
    draw=blue!40!black!90, very thick,
    text width=10em, 
    minimum height=1.5em, 
    text centered, 
    on chain},
  every join/.style={->, thick,shorten <=1pt},
  decoration={brace},
  tuborg/.style={decorate},
  tubnode/.style={midway, right=2pt},
}

\mode<presentation>

\usetheme{Warsaw}

\definecolor{Red}{rgb}{1,0,0}
\definecolor{Blue}{rgb}{0,0,1}
\definecolor{Green}{rgb}{0,1,0}
\definecolor{magenta}{rgb}{1,0,.6}
\definecolor{lightblue}{rgb}{0,.5,1}
\definecolor{lightpurple}{rgb}{.6,.4,1}
\definecolor{gold}{rgb}{.6,.5,0}
\definecolor{orange}{rgb}{1,0.4,0}
\definecolor{hotpink}{rgb}{1,0,0.5}
\definecolor{newcolor2}{rgb}{.5,.3,.5}
\definecolor{newcolor}{rgb}{0,.3,1}
\definecolor{newcolor3}{rgb}{1,0,.35}
\definecolor{darkgreen1}{rgb}{0, .35, 0}
\definecolor{darkgreen}{rgb}{0, .6, 0}
\definecolor{darkred}{rgb}{.75,0,0}

\xdefinecolor{olive}{cmyk}{0.64,0,0.95,0.4}
\xdefinecolor{purpleish}{cmyk}{0.75,0.75,0,0}

\useoutertheme[subsection=false]{smoothbars}


\title{Ruby: массивы}
\author{Информатика \\ 10-11 классы}

%\usecolortheme{dolphin}


\begin{document}
%%титульная страница
\maketitle
%% основные моменты

\section{Базовые сведения}

\subsection{Введение}
\begin{frame}
  \frametitle{Введение}
  \begin{itemize}
    \item Одним из базовых типов переменных является \emph{массив}. Ранее мы рассматривали переменные типа число (integer, float), немного строки (string).
    \item (Формально): Массив --- ряд переменных, доступ к которым определён по индексу.
    \item (Упрощённо): Массив --- группа переменных, пронумеранных начиная с нуля, хранящихся в единой переменной.
    \item Массивы позволяют упростить доступ к однотипным данным.
    \item Например, в виде массива легко хранить температуру за 30 последних дней.
    \item Массивы бывают двумерными --- когда элементы массива сами являются массивами.
  \end{itemize}
\end{frame}

\subsection{Введение}
\begin{frame}[fragile]
  \frametitle{Табличная форма}
  \begin{columns}[lc]
  \column{1.0in}
    \begin{tabular}{|c|c|}
    \hline
    Ключ & Значение\\
    \hline
    0 & 5.3\\
    \hline
    1 & 4.2\\
    \hline
    2 & 2.0\\
    \hline
    3 & -0.8\\
    \hline
    4 & 1.5\\
    \hline
    \end{tabular}
    
  \column{2.8in}
    \begin{itemize}
      \item Рассмотрим массив, состоящий из 5 элементов (см. таблицу слева)
      \item В ruby такой массив записывается следующим образом:
    \end{itemize}
      \begin{lstlisting}[language=ruby,basicstyle=\footnotesize,label=ruby1,caption=Создание массива]
        arr = [5.3, 4.2, 2.0, -0.8, 1.5]
      \end{lstlisting}
    \begin{itemize}
      \item где \rm{arr} --- название массива.
      \item Чтобы вывести на экран, например, элемент с ключом 3 достаточно написать \textbf{puts arr[3]}:
    \end{itemize}
  \end{columns}
\end{frame}

\subsection{Создание массива}
\begin{frame}[fragile]
  \frametitle{Создание массива}
  
  \begin{itemize}
    \item Зададим массив, состоящий из 6 последовательных натуральных чисел.
  \end{itemize}
    \begin{lstlisting}[language=ruby,basicstyle=\footnotesize,label=ruby2,caption=Способы создания массива]
    arr = [1,2,3,4,5,6]
    
    arr = Array.new
    arr[0] = 1
    arr[1] = 2
    arr[]  = 3
    ...
    
    arr = (1..6).to_a
    \end{lstlisting}

  \begin{itemize}
    \item 1 способ --- обычный, 2 --- ручной (обратите внимание, что ключ при добавлении элемента в конец указывать необязательно), 3 --- через диапазон.
  \end{itemize}
  
\end{frame}

\section{Методы}
\subsection{Методы}
\begin{frame}[fragile]
  \frametitle{Методы}
  
  \begin{itemize}
    \item Чтобы изменить массив, к нему нужно применить так называемый \emph{метод}.
    \item \emph{Метод} --- это то, что меняет переменную по заданному правилу.
    \item Например, метод \textbf{sort} сортирует массив.
    \item Для разных типов переменных могут быть разные методы.
    \item Чтобы применить к переменной \textbf{var} метод \textbf{method} достаточно написать:
  \end{itemize}
  \begin{lstlisting}[language=ruby,basicstyle=\footnotesize,label=ruby3,caption=Применение метода]
    var.method
  \end{lstlisting}
  \begin{itemize}
    \item Обратите внимание! Обычный метод не меняет саму переменную. Хотите изменить --- присвойте результат другой переменной или этой же самой (var = var.method)
  \end{itemize}
  
\end{frame}

\subsection{Базовые методы}
\begin{frame}[fragile]
  \frametitle{Методы работы с массивом}
		\begin{itemize}
  	  \item Рассмотрим массив arr = [6,3,5,1,2,4].
		\end{itemize}
		
		\begin{tabular}{|l|l|c|}
		\hline
		\centering{\textbf{Метод}} & \centering{\textbf{Описание}} & \textbf{Результат}\\
		\hline
		arr.size & количество элементов & 6 \\
		\hline
		arr.min & минимальный элемент & 1 \\
		\hline
		arr.max & максимальный элемент & 6 \\
		\hline
		arr.sort & сортировка & [1,2,3,4,5,6] \\
		\hline
		arr.reverse & перевернуть & [4,2,1,5,3,6] \\
		\hline
		arr.sort.reverse & отсортировать и перевернуть & [6,5,4,3,2,1] \\
		\hline
		arr.include?(6) & есть ли в массиве элемент 6 & true \\
		\hline
		arr.empty? & пуст ли массив & false \\
		\hline
		arr.any? & есть ли хоть один элемент & true \\
		\hline
		arr.delete(6) & удалить элемент 6* & [3,5,1,2,4] \\
		\hline
		arr.delete\_at(1) & удалить элемент с ключом 1 & [6,5,1,2,4] \\
		\hline
		\end{tabular}
		
\end{frame}

\section{Задачи}

\subsection{Сумма}
\begin{frame}[fragile]
  \frametitle{Сумма}
		\begin{itemize}
  		\item Допустим дана средняя температура за 5 дней: arr = [5.3, 2.1, 5.2, 1.8, -0.2].
  		\item Как вычислить сумму?
  	\end{itemize}
  	\scriptsize{
  		\begin{lstlisting}[language=ruby,basicstyle=\footnotesize,label=ruby5,caption=Сумма]
  		  arr = [5.3, 2.1, 5.2, 1.8, -0.2]
  		  sum = arr.inject(0){ |res, elem| res+elem }
      \end{lstlisting}}
  	
  	\begin{figure}
    \centering
    \begin{tikzpicture}[node distance=2.0em]
    \tikzset{
        every node/.style={scale=0.7},
        action/.style={rectangle,draw=black, top color=white, bottom color=yellow!50,thin, inner sep=0.25em, minimum size=0.6em, text centered},
        input/.style={ellipse,draw=black, top color=white, bottom color=yellow!50,thin, inner sep=0.25em, minimum size=0.6em, text centered},
        condition/.style={diamond,draw=black, top color=white, bottom color=yellow!50,thin, inner sep=0.25em, minimum size=0.6em, text centered},
        block/.style={chamfered rectangle,draw=black, top color=white, bottom color=yellow!50,thin, inner sep=0.25em, minimum size=0.6em, text centered},
        myarrow/.style={draw},
    }
    
    \node[action] (AuxNode01) {res=0};
    \node[block, below=1.0em of AuxNode01] (item1) {по всем элементам массива};
    \node[action, below right=1.0em of item1] (item2) {res=res+elem};
    \node (AuxNode02) [text width=0em, below=1.0em of item2] {};
    \node (AuxNode03) [text width=0em, left=5.5em of AuxNode02] {};
    \node (AuxNode04) [text width=0em, left=7.0em of AuxNode03] {};
    \node (AuxNode05) [text width=0em, left=1.2em of item1] {};
    \node (AuxNode06) [text width=0em, below=1.0em of AuxNode03] {};

    \path[myarrow] (AuxNode01.south) -- (item1.north);   
    \path[myarrow] (item1.east) -| (item2.north);   
    \path[myarrow] (item2.south) -- (AuxNode02.center);   
    \path[myarrow] (AuxNode02.center) -- (AuxNode03.center);   
    \path[myarrow] (AuxNode03.center) -- (AuxNode04.center);   
    \path[myarrow] (AuxNode04.center) -- (AuxNode05.center);   
    \path[myarrow] (AuxNode05.center) -- (item1.west);   
    \path[myarrow] (AuxNode03.center) -- (AuxNode06.center);   

    \end{tikzpicture} 
    \end{figure}
\end{frame}

\subsection{Inject}
\begin{frame}[fragile]
  \frametitle{Разбор метода inject}

  \begin{lstlisting}[language=ruby,basicstyle=\footnotesize,label=ruby6,caption=Общий вид inject]
    res = arr.inject(start){|result, element| expression }
  \end{lstlisting}
  
  \begin{itemize}
    \item start --- чему изначально равна переменная result. Например, если внутри inject вы будете умножать, то нельзя делать переменную равной нулю, так как ноль умножить на любое число --- ноль.
    \item result --- переменная, в которую записывается результат.
    \item element --- текущий элемент массива (переменная меняется с каждой итерацией).
    \item expression --- выражение для переменных result и element.
    \item \textbf{Обратите внимание!} На этом слайде названия переменных и выражение даны в виде схемы! Реальный пример --- на предыдущем слайде.
  \end{itemize}
  
\end{frame}

\subsection{Задания}
\begin{frame}
  \frametitle{Задания}
  \begin{itemize}
    \item Напишите программу, вычисляющую с помощью inject произведение элементов массива (массив можно задать любой).
    \item Напишите программу, вычисляющую среднее арифметическое и среднее геометрическое элементов массива.
  \end{itemize}
\end{frame}

\subsection{Поиск}
\begin{frame}[fragile]
  \frametitle{Поиск элементов}

  \begin{itemize}
    \item В ряде задач нам нужно извлечь из массива определённые элементы.
		\item Допустим дана средняя температура за 5 дней: arr = [5.3, 2.1, 1.2, -0.8, -0.2].
    \item Вычислим, сколько дней была отрицательная температура.
  \end{itemize}

  \begin{lstlisting}[language=ruby,basicstyle=\footnotesize,label=ruby7,caption=Метод find\_all]
    arr  = [5.3, 2.1, 1.2, -0.8, -0.2]
    arr_neg = arr.find_all{|elem| (elem < 0) }
    puts arr_neg.size
    
    puts [5.3, 2.1, 1.2, -0.8, -0.2].
         find_all{|elem| (elem > 0) }.size
  \end{lstlisting}
  
\end{frame}

\subsection{Разбор метода find}
\begin{frame}[fragile]
  \frametitle{Разбор метода find\_all}
  
  \begin{lstlisting}[language=ruby,basicstyle=\footnotesize,label=ruby8,caption=Общий вид find\_all]
    arr_res = arr.find_all{ |element| condition }
  \end{lstlisting}
  
  \begin{itemize}
    \item element --- текущий (рассматриваемый) элемент массива (переменная меняется с каждой итерацией).
    \item condition --- логическое выражение или несколько выражений, связанных логическими операциями (конъюнкция, дизъюнкция, отрицание).
    \item Метод извлекает из массива \textbf{arr} все элементы, удовлетворяющие условию \textbf{condition}, и записывает результат в массив \textbf{arr\_res}.
  \end{itemize}
  
\end{frame}

\subsection{Задания}
\begin{frame}
  \frametitle{Задания}
  \begin{itemize}
    \item Дан массив из 10 целых чисел (любых). Вывести на экран все элементы массива, меньшие нуля и делящиеся на три.
    \item Дан массив из 10 целых чисел (любых). Вывести на экран сумму все чётных положительных элементов массива.
  \end{itemize}
\end{frame}

\subsection{map}
\begin{frame}[fragile]
  \frametitle{Изменение элементов массива}

  \begin{itemize}
    \item Предположим, у нас есть массив цен на нефть в долларах США за последние 5 дней.
    \item Как перевести все цены в рубли, зная курс рубля по отношению к доллару?
  \end{itemize}

  \begin{lstlisting}[language=ruby,basicstyle=\footnotesize,label=ruby9,caption=Метод map]
    arr  = [99.23, 101.42, 99.87, 96.49, 95.11]
    usd_to_rub = 31.23
    arr_in_rub = arr.map{|elem| elem*usd_to_rub}
    puts arr_in_rub
  \end{lstlisting}
  
\end{frame}

\subsection{Разбор метода map}
\begin{frame}[fragile]
  \frametitle{Разбор метода map}
  
  \begin{lstlisting}[language=ruby,basicstyle=\footnotesize,label=ruby10,caption=Общий вид map]
    arr_res = arr.map{|element| expression }
  \end{lstlisting}
  
  \begin{itemize}
    \item element --- текущий (рассматриваемый) элемент массива (переменная меняется с каждой итерацией).
    \item expression --- выражение, показывающее, как надо менять элемент массива.
    \item Метод проходит по всему массиву \textbf{arr} и меняет каждый элемент в соответствии с выражением \textbf{expression}. Результат записывается в массив \textbf{arr\_res}.
  \end{itemize}
  
\end{frame}

\subsection{Задания}
\begin{frame}
  \frametitle{Задания}
  \begin{itemize}
    \item Дан массив температуры за последние 10 дней (любые разумные числа) в градусах по Цельсию. Вывести на экран температуру на каждый день в градусах по Фаренгейту и по Кельвину.
  \end{itemize}
\end{frame}

\section{References}
\subsection{References}
\begin{frame}[fragile]
  \frametitle{References}
  \begin{itemize}
    \item Все презентации доступны на http://school.smirik.ru!
    \item Вопросы, предложения, д/з: smirik@gmail.com
  \end{itemize}
\end{frame}

\end{document}