\documentclass[compress,red]{beamer}
\usepackage[utf8]{inputenc}
\usepackage{ucs}
\usepackage{amsmath}
\usepackage{amsfonts}
\usepackage{amssymb}
\usepackage[russian]{babel}
\usepackage{graphicx}
\usepackage{wrapfig}

\usepackage{tikz}
\usepackage{verbatim}

\usepackage{color}
\usepackage{xcolor}
\usepackage{listings}

\usepackage{caption}

\lstset{
language=ruby,
extendedchars=\true,
inputencoding=utf8x,
commentstyle=\itshape,
stringstyle=\bf,
belowcaptionskip=5pt }


\DeclareCaptionFont{white}{\color{white}}
\DeclareCaptionFormat{listing}{\colorbox{gray}{\parbox{\textwidth}{#1#2#3}}}
\captionsetup[lstlisting]{format=listing,labelfont=white,textfont=white}

\usetikzlibrary{calc,trees,positioning,arrows,chains,shapes.geometric,%
    decorations.pathreplacing,decorations.pathmorphing,shapes,%
    matrix,shapes.symbols}

\tikzset{
>=stealth',
  punktchain/.style={
    rectangle, 
    rounded corners, 
    % fill=black!10,
    draw=black, very thick,
    text width=10em, 
    minimum height=3em, 
    text centered, 
    on chain},
  line/.style={draw, thick, <-},
  element/.style={
    tape,
    top color=white,
    bottom color=blue!50!black!60!,
    minimum width=8em,
    draw=blue!40!black!90, very thick,
    text width=10em, 
    minimum height=1.5em, 
    text centered, 
    on chain},
  every join/.style={->, thick,shorten <=1pt},
  decoration={brace},
  tuborg/.style={decorate},
  tubnode/.style={midway, right=2pt},
}

\mode<presentation>

\usetheme{Warsaw}

\definecolor{Red}{rgb}{1,0,0}
\definecolor{Blue}{rgb}{0,0,1}
\definecolor{Green}{rgb}{0,1,0}
\definecolor{magenta}{rgb}{1,0,.6}
\definecolor{lightblue}{rgb}{0,.5,1}
\definecolor{lightpurple}{rgb}{.6,.4,1}
\definecolor{gold}{rgb}{.6,.5,0}
\definecolor{orange}{rgb}{1,0.4,0}
\definecolor{hotpink}{rgb}{1,0,0.5}
\definecolor{newcolor2}{rgb}{.5,.3,.5}
\definecolor{newcolor}{rgb}{0,.3,1}
\definecolor{newcolor3}{rgb}{1,0,.35}
\definecolor{darkgreen1}{rgb}{0, .35, 0}
\definecolor{darkgreen}{rgb}{0, .6, 0}
\definecolor{darkred}{rgb}{.75,0,0}

\xdefinecolor{olive}{cmyk}{0.64,0,0.95,0.4}
\xdefinecolor{purpleish}{cmyk}{0.75,0.75,0,0}

\useoutertheme[subsection=false]{smoothbars}

\title{Статические свойства и методы и сокрытие данных}
\author{Информатика \\ 10-11 классы}

%\usecolortheme{dolphin}


\begin{document}
%%титульная страница
\maketitle
%% основные моменты

\section{Сокрытие данных}

\subsection{Сокрытие данных}
\begin{frame}[fragile]
  \frametitle{Сокрытие данных}
  \begin{itemize}
    \item Одним из важнейших свойств инкапсуляции является \emph{сокрытие методов}.
    \item Пример (rubydev): для пояснения термина нам помогут самые обычные хоббиты.
    \item Они дружественны и открыты для других, но и у них есть также личная жизнь. 
    \item Многое из их личной жизни доступно для других хоббитов так как в своем хоббитовом царстве они могут бродить по жилищам друг друга. 
    \item Конечно, каждый хоббит имеет что-то очень личное, такое, что знает один лишь он и никому и никогда не раскроет своей тайны. 
    \item Такие особенности устройства поведения хоббитов очень похожи на методы в Ruby. 
    \item Давайте представим каждого хоббита как объект.
  \end{itemize}
\end{frame}

\subsection{Класс Хоббит}
\begin{frame}[fragile]
  \frametitle{Класс Хоббит}
  \scriptsize{
  \begin{lstlisting}[language=ruby,basicstyle=\footnotesize,label=ruby1,caption=Класс Хоббит]
    class Hobbit
      def initialize(name, rooms, has_ring)
        @name, @rooms, @has_ring = name, rooms, has_ring
      end

      def name
        @name
      end

      def name_of(hobbit)
        hobbit.name
      end

      def rooms_of(hobbit)
        hobbit.rooms
      end
  \end{lstlisting}
  }
\end{frame}

\subsection{Класс Хоббит 2}
\begin{frame}[fragile]
  \frametitle{Класс Хоббит}
  \scriptsize{
  \begin{lstlisting}[language=ruby,basicstyle=\footnotesize,label=ruby2,caption=Класс Хоббит]
      def hobbit_has_ring?(hobbit)
        hobbit.has_ring?
      end

      protected

      def rooms
        @rooms
      end

      private

      def has_ring?
        @has_ring
      end
    end
  \end{lstlisting}
  }
\end{frame}

\subsection{Класс Хоббит описание}
\begin{frame}[fragile]
  \frametitle{Что мы получили?}
  \begin{itemize}
    \item Итого, в нашем классе Хоббит есть 5 методов.
    \item Три из них расположены в публичной (public) зоне, которая всегда идёт по умолчанию: name\_of, rooms\_of, hobbit\_has\_ring?
    \item Один метод является защищённым (protected): rooms.
    \item Ещё один метод --- приватным (private): has\_ring?
  \end{itemize}
\end{frame}

\subsection{Класс Хоббит использование}
\begin{frame}[fragile]
  \frametitle{Используем хоббитов}
  \begin{itemize}
    \item Создадим пару хоббитов и выведем их имена.
  \end{itemize}
  \scriptsize{
  \begin{lstlisting}[language=ruby,basicstyle=\footnotesize,label=ruby3,caption=Использование Хоббита]
    frodo = Hobbit.new(‘Frodo’, 3, true)
    samwise = Hobbit.new(‘Samwise’, 2, false)
    
    puts frodo.name
    puts samwise.name
  \end{lstlisting}
  }
\end{frame}

\subsection{Защищённые методы}
\begin{frame}[fragile]
  \frametitle{Защищённые методы}
  \begin{itemize}
    \item А если мы хотим узнать количество комнат у хоббита?
    \item Простой вызов \textbf{frodo.rooms} вызовет ошибку, так как метод rooms является защищённым.
    \item Защищённые методы можно вызвать только изнутри объекта. При вызове снаружи возникает ошибка.
    \item На нашем примере: хоббиты скрывают количество комнат в своих домах от других живых существ, но между собой у них в этом секрета нет.
    \item Любой хоббит может узнать количество комнат себя или другого хоббита.
  \end{itemize}
\end{frame}

\subsection{Количество комнат}
\begin{frame}[fragile]
  \frametitle{Количество комнат}
  \scriptsize{
  \begin{lstlisting}[language=ruby,basicstyle=\footnotesize,label=ruby4,caption=Количество комнат]
  frodo   = Hobbit.new(‘Frodo’, 3, true)
  samwise = Hobbit.new(‘Samwise’, 2, false)
  
  puts frodo.rooms_of(frodo)
  puts frodo.rooms_of(samwise)
    
  \end{lstlisting}
  }
  
\end{frame}

\subsection{Приватные методы}
\begin{frame}[fragile]
  \frametitle{А что насчёт кольца?}
  \begin{itemize}
    \item Кольцо расположено в приватной области.
    \item При попытке вызвать метод has\_ring или hobbit\_has\_ring возникает ошибка:
  \end{itemize}
  \scriptsize{
  \begin{lstlisting}[language=ruby,basicstyle=\footnotesize,label=ruby5,caption=Ошибка при вызове]
    ...
    frodo.has_ring?
    frodo.hobbit_has_ring?(frodo)  
    frodo.hobbit_has_ring?(samwise)  
  \end{lstlisting}
  }
\end{frame}

\subsection{Пояснение}
\begin{frame}[fragile]
  \frametitle{Приватные методы}
  \begin{itemize}
    \item Доступ к приватным методам возможен только из самого объекта.
    \item Извне ни к своим, ни к чужим методам доступ получить нельзя.
    \item Условно говоря, приватные свойства и методы --- это та информация, которую знает только один конкретный экземпляр класса.
  \end{itemize}
\end{frame}

\subsection{Статические свойства и методы}
\begin{frame}[fragile]
  \frametitle{Статические свойства и методы}
  \begin{itemize}
    \item В обычном случае мы вызываем методы у объектов.
    \item То есть, процедура такая: написать класс, создать объект, вызвать метод.
    \item Однако методы могут быть не только у объекта, но и у класса в целом.
    \item По своей логике они напоминают обычные функции, в названии которых дополнительно встречается приставка с именем класса.
  \end{itemize}
\end{frame}

\subsection{Пример}
\begin{frame}[fragile]
  \frametitle{Пример}
  \begin{itemize}
    \item 
  \end{itemize}
  \scriptsize{
  \begin{lstlisting}[language=ruby,basicstyle=\footnotesize,label=ruby6,caption=Пример статического метода]
    class Hobbit

      def self.location
        return "Shire"
      end

    end

    puts Hobbit.location
  \end{lstlisting}
  }
  
\end{frame}

\subsection{Статические переменные}
\begin{frame}[fragile]
  \frametitle{Статические переменные}
  \begin{itemize}
    \item Статические переменные --- это переменные, которые существуют у класса в единственном экземпляре.
    \item То есть, разные экземпляры класса будут иметь одну и ту же переменную.
  \end{itemize}
\end{frame}

\subsection{Пример программы}
\begin{frame}[fragile]
  \frametitle{Пример программы}
  \scriptsize{
  \begin{lstlisting}[language=ruby,basicstyle=\footnotesize,label=ruby7,caption=Статическое свойство]
  class Hobbit
    attr_accessor :number
    @@number = 0
    def initialize
      @@number+=1
    end
    def show
      puts "#{@@number}"
    end
  end
  frodo = Hobbit.new
  frodo.show
  samwise = Hobbit.new
  samwise.show
  frodo.show
  \end{lstlisting}
  }
\end{frame}

\subsection{Задание}
\begin{frame}[fragile]
  \frametitle{Задание}
  \begin{itemize}
    \item Написать эссе на тему: Что такое Синглтон (Singleton). Реализовать на синглтон на ruby и продемонстрировать работу.
  \end{itemize}
\end{frame}

\section{References}
\subsection{References}
\begin{frame}[fragile]
  \frametitle{References}
  \begin{itemize}
    \item При подготовке данного материала использовались сайты: http://ru.wikibooks.org/wiki/Ruby, http://rubydev.ru, http://en.wikipedia.org, http://ruby-lang.org.
    \item Все презентации доступны на http://school.smirik.ru!
    \item Вопросы, предложения, д/з: smirik@gmail.com
  \end{itemize}
\end{frame}

\end{document}