\documentclass[compress,red]{beamer}
\usepackage[utf8]{inputenc}
\usepackage{ucs}
\usepackage{amsmath}
\usepackage{amsfonts}
\usepackage{amssymb}
\usepackage[russian]{babel}
\usepackage{graphicx}
\usepackage{wrapfig}

\usepackage{tikz}
\usepackage{verbatim}

\usepackage{color}
\usepackage{xcolor}
\usepackage{listings}

\usepackage{caption}

\lstset{
language=ruby,
extendedchars=\true,
inputencoding=utf8x,
commentstyle=\itshape,
stringstyle=\bf,
belowcaptionskip=5pt }


\DeclareCaptionFont{white}{\color{white}}
\DeclareCaptionFormat{listing}{\colorbox{gray}{\parbox{\textwidth}{#1#2#3}}}
\captionsetup[lstlisting]{format=listing,labelfont=white,textfont=white}

\usetikzlibrary{calc,trees,positioning,arrows,chains,shapes.geometric,%
    decorations.pathreplacing,decorations.pathmorphing,shapes,%
    matrix,shapes.symbols}

\tikzset{
>=stealth',
  punktchain/.style={
    rectangle, 
    rounded corners, 
    % fill=black!10,
    draw=black, very thick,
    text width=10em, 
    minimum height=3em, 
    text centered, 
    on chain},
  line/.style={draw, thick, <-},
  element/.style={
    tape,
    top color=white,
    bottom color=blue!50!black!60!,
    minimum width=8em,
    draw=blue!40!black!90, very thick,
    text width=10em, 
    minimum height=1.5em, 
    text centered, 
    on chain},
  every join/.style={->, thick,shorten <=1pt},
  decoration={brace},
  tuborg/.style={decorate},
  tubnode/.style={midway, right=2pt},
}

\mode<presentation>

\usetheme{Warsaw}

\definecolor{Red}{rgb}{1,0,0}
\definecolor{Blue}{rgb}{0,0,1}
\definecolor{Green}{rgb}{0,1,0}
\definecolor{magenta}{rgb}{1,0,.6}
\definecolor{lightblue}{rgb}{0,.5,1}
\definecolor{lightpurple}{rgb}{.6,.4,1}
\definecolor{gold}{rgb}{.6,.5,0}
\definecolor{orange}{rgb}{1,0.4,0}
\definecolor{hotpink}{rgb}{1,0,0.5}
\definecolor{newcolor2}{rgb}{.5,.3,.5}
\definecolor{newcolor}{rgb}{0,.3,1}
\definecolor{newcolor3}{rgb}{1,0,.35}
\definecolor{darkgreen1}{rgb}{0, .35, 0}
\definecolor{darkgreen}{rgb}{0, .6, 0}
\definecolor{darkred}{rgb}{.75,0,0}

\xdefinecolor{olive}{cmyk}{0.64,0,0.95,0.4}
\xdefinecolor{purpleish}{cmyk}{0.75,0.75,0,0}

\useoutertheme[subsection=false]{smoothbars}

\title{Математическая логика}

%\usecolortheme{dolphin}


\begin{document}
%%титульная страница
\maketitle
%% основные моменты

\section{Основные операции}

\subsection{Три базовые операции}
\begin{frame}[fragile]
  \frametitle{Три основные логические операции}
  
  \begin{tabular}{|l|c|c|l|l|}
    \hline
    Название & Обозначение  & Альт. & Соответствие & Знач. \\
    \hline
    Конъюнкция & $\&$ & $\wedge$ & умножение & и \\
    \hline
    Дизъюнкция & $\vee$ &  & сложение & или \\
    \hline
    Отрицание & $\neg$ & \bar{ } & отрицание & не \\
    \hline
  \end{tabular}
  
  \begin{itemize}
    \item Полезно запомнить соответствия --- это помогает определить порядок выполнения операций: отрицание, конъюнкция, дизъюнкция.
  \end{itemize}
\end{frame}

\subsection{Запомним конъюнкцию}
\begin{frame}
  \begin{center}
    \Huge{$\&$}
  \end{center}
\end{frame}

\subsection{Запомним отрицание}
\begin{frame}
  \begin{center}
    \Huge{$\neg$}
  \end{center}
\end{frame}

\subsection{Запомним дизъюнкцию}
\begin{frame}
  \begin{center}
    \Huge{$\vee$}
  \end{center}
\end{frame}

\subsection{Порядок выполнения 1}
\begin{frame}
  \begin{center}
    \Huge{$A \vee {\neg} B \& C$}
  \end{center}
  \begin{center}
    \Large{Какой порядок выполнения?}
  \end{center}
\end{frame}

\subsection{Порядок выполнения 2}
\begin{frame}
  \begin{center}
    \Huge{${\neg} A \& {\neg} B \vee C \& ({\neg} D \vee E)$}
  \end{center}
  \begin{center}
    \Large{Какой порядок выполнения?}
  \end{center}
\end{frame}

\section{Таблицы истинности}
\subsection{Таблицы истинности}
\begin{frame}
  \begin{center}
    \Huge{Таблицы истинности}
  \end{center}
\end{frame}

\subsection{Таблица истинности для конъюнкции}
\begin{frame}[fragile]
  \frametitle{Таблица истинности для конъюнкции}
  \begin{itemize}
    \item Рассмотрим, чему может быть равно высказывание A \& B:
  \end{itemize}
  \begin{center}
    \begin{tabular}{|c|c|c|}
      \hline
      A & B & A \& B \\
      \hline
      И & И & И \\
      \hline
      И & Л & Л \\
      \hline
      Л & И & Л \\
      \hline
      Л & Л & Л \\
      \hline
    \end{tabular}
  \end{center}
\end{frame}

\subsection{Таблица истинности для дизъюнкции}
\begin{frame}[fragile]
  \frametitle{Таблица истинности для дизъюнкции}
  \begin{itemize}
    \item Рассмотрим, чему может быть равно высказывание $A \vee B$:
  \end{itemize}
  \begin{center}
    \begin{tabular}{|c|c|c|}
      \hline
      A & B & $A \vee B$ \\
      \hline
      И & И & И \\
      \hline
      И & Л & И \\
      \hline
      Л & И & И \\
      \hline
      Л & Л & Л \\
      \hline
    \end{tabular}
  \end{center}
\end{frame}

\subsection{Таблица истинности для отрицания}
\begin{frame}[fragile]
  \frametitle{Таблица истинности для отрицания}
  \begin{itemize}
    \item Рассмотрим, чему может быть равно высказывание ${\neg} A$:
  \end{itemize}
  \begin{center}
    \begin{tabular}{|c|c|}
      \hline
      A & ${\neg} A$ \\
      \hline
      И & Л \\
      \hline
      Л & И \\
      \hline
    \end{tabular}
  \end{center}
\end{frame}

\subsection{Правила запоминания}
\begin{frame}[fragile]
  \frametitle{Правила для запоминания / понимания}
  \begin{itemize}
    \item Конъюнкция требует, чтобы оба условия были истинны: \textbf{и} то, \textbf{и} другое.
    \item Дизъюнкции достаточно, чтобы одно из условий выполнялось: \textbf{или} первое, \textbf{или} второе (\textbf{или} оба вместе).
    \item Отрицание меняет значение на противоположное.
  \end{itemize}
\end{frame}

\subsection{Пример построения таблицы истинности}
\begin{frame}[fragile]
  \frametitle{Пример построения таблицы истинности}
  \begin{itemize}
    \item Построим таблицу истинности для высказывания $A$~\&~$B\vee{\neg}C$.
    \item Построение заключается в переборе всех возможных вариантов.
  \end{itemize}
  \begin{center}
  \begin{tabular}{|c|c|c|c|}
    \hline
    A & B & C & Итог \\
    \hline
    И & И & И &  \\
    \hline
    И & И & Л &  \\
    \hline
    И & Л & И &  \\
    \hline
    Л & И & И &  \\
    \hline
    И & Л & Л &  \\
    \hline
    Л & И & Л &  \\
    \hline
    Л & Л & И &  \\
    \hline
    Л & Л & Л &  \\
    \hline
  \end{tabular}
  \end{center}
\end{frame}

\subsection{Пример построения таблицы истинности 2}
\begin{frame}[fragile]
  \frametitle{Пример построения таблицы истинности}
  \begin{itemize}
    \item Посчитаем первую строку, подставив значения A, B, C:
    \item $\rm{A}\&\rm{B}\vee{\neg}\rm{C} =$ И$\&$И$\vee{\neg}$И = И$\&$И$\vee$Л = И $\vee$ Л $=$ И
  \end{itemize}
  \begin{center}
  \begin{tabular}{|c|c|c|c|}
    \hline
    A & B & C & Итог \\
    \hline
    И & И & И & И \\
    \hline
    И & И & Л &  \\
    \hline
    И & Л & И &  \\
    \hline
    Л & И & И &  \\
    \hline
    И & Л & Л &  \\
    \hline
    Л & И & Л &  \\
    \hline
    Л & Л & И &  \\
    \hline
    Л & Л & Л &  \\
    \hline
  \end{tabular}
  \end{center}
\end{frame}

\subsection{Пример построения таблицы истинности 3}
\begin{frame}[fragile]
  \frametitle{Пример построения таблицы истинности}
  \begin{itemize}
    \item Аналогично рассчитываем для каждой оставшейся строки.
  \end{itemize}
  \begin{center}
  \begin{tabular}{|c|c|c|c|}
    \hline
    A & B & C & Итог \\
    \hline
    И & И & И & И \\
    \hline
    И & И & Л & И \\
    \hline
    И & Л & И & Л \\
    \hline
    Л & И & И & Л \\
    \hline
    И & Л & Л & И \\
    \hline
    Л & И & Л & И \\
    \hline
    Л & Л & И & Л \\
    \hline
    Л & Л & Л & И \\
    \hline
  \end{tabular}
  \end{center}
\end{frame}

\subsection{Задачи на таблицы истинности}
\begin{frame}[fragile]
  \frametitle{Задачи}
  \begin{itemize}
    \item Построить таблицу истинности для следующих высказываний:
      \begin{enumerate}
        \item A & ${\neg}$ B $\vee$ ${\neg {\neg}}$A
        \item ${\neg} ($A$\&{\neg}$B$)\&C $
        \item A & ${\neg}A$
        \item A $\vee$ ${\neg}A$
      \end{enumerate}
  \end{itemize}
\end{frame}

\section{Законы логики}
\subsection{Законы логики}
\begin{frame}
  \begin{center}
    \Huge{Законы логики}
  \end{center}
\end{frame}

\subsection{Законы логики список}
\begin{frame}[fragile]
  \frametitle{Основные законы логики}
  \begin{itemize}
    \item Как и в математике, в логике есть свои законы.
    \item Во многом, они похожи на математические.
    
    \begin{tabular}{|l|l|l|}
      \hline
      Название & Закон \\
      \hline
      Переместительный & A$\&$B~=~B$\&$A \\ 
                       & A$\vee$B~=~B$\vee$A \\
      \hline
      Сочетательный & A$\&$(B$\&$C)~=~A$\&$(B$\&$C) \\
                    & A$\vee$(B$\vee$C)~=~A$\vee$(B$\vee$C) \\
      \hline
      Распределительный & A$\&$(B$\vee$C)~=~(A$\&$B)$\vee$(A$\&$C) \\
                    & A$\vee$(B$\&$C)~=~(A$\vee$B)$\&$(A$\vee$C) \\
      \hline
      Правила де Моргана & ${\neg}$(A$\&$B) = ${\neg}$A$\vee$${\neg}$B \\
                         & ${\neg}$(A$\vee$B) = ${\neg}$A$\&$${\neg}$B \\
      \hline
    \end{tabular}
  \end{itemize}
\end{frame}

\subsection{Доказательства}
\begin{frame}
  \begin{center}
    \Large{В логике можно доказывать законы}
  \end{center}
  \begin{center}
    \Huge{Но как?}
  \end{center}
\end{frame}

\subsection{Доказательство 2}
\begin{frame}[fragile]
  \frametitle{Доказательство правила де Моргана}
  \begin{itemize}
    \item В логике два высказывания называются равными, если совпадают их таблицы истинности.
    \item Для доказательства одного из правил де Моргана построим таблицы истинности для левых и правых частей равенства.
    \item Если таблицы совпадут, значит, формула верна:


    \begin{columns}[c]
    \column{1.5in}
      \begin{tabular}{|c|c|c|}
        \hline
        A & B & ${\neg}$(A$\&$B) \\
        \hline
        И & И & Л \\
        \hline
        И & Л & И \\
        \hline
        Л & И & И \\
        \hline
        Л & Л & И \\
        \hline
      \end{tabular}
    \column{1.5in}
      \begin{tabular}{|c|c|c|}
        \hline
        A & B & ${\neg}$A$\vee$${\neg}$B \\
        \hline
        И & И & Л \\
        \hline
        И & Л & И \\
        \hline
        Л & И & И \\
        \hline
        Л & Л & И \\
        \hline
      \end{tabular}
    \end{columns}
    
  \end{itemize}
\end{frame}

\subsection{Задачи по законам}
\begin{frame}[fragile]
  \frametitle{Задачи}
  \begin{itemize}
    \item Докажите переместительный закон для конъюнкции.
    \item Докажите распределительный закон конъюнкции относительно дизъюнкции (первый).
  \end{itemize}
\end{frame}

\subsection{Малые законы}
\begin{frame}[fragile]
  \frametitle{Остальные законы}
  \begin{center}
  \begin{tabular}{|l|l|l|}
    \hline
    Название & Закон \\
    \hline
    Двойного отрицания & ${\neg{\neg}}$A = A \\ 
    \hline
    Противоречия & A$\&{\neg}$A = Л \\
                  & A$\vee{\neg}$A = И \\
    \hline
    Повторения & A$\&$A = A \\
               & A$\vee$A = A \\
    \hline
    Просто закон & A$\&$И = A \\
              & A$\vee$И = И \\
    \hline
    Просто закон 2 & A$\&$Л = Л \\
              & A$\vee$Л = A \\
    \hline
  \end{tabular}
  \end{center}
\end{frame}

\subsection{Задачи на законы}
\begin{frame}[fragile]
  \frametitle{Задачи}
  \begin{itemize}
    \item Пример с раскрытием скобок: 
    \item A$\&{\neg}$(B$\vee$C)~=~A$\&$(${\neg}$B$\&$${\neg}$C)~=~A$\&{\neg}$B$\&$${\neg}$C
    \item \textbf{Задача.} Раскрыть скобки у следующих выражений:
      \begin{enumerate}
        \item ${\neg} ($A$\&{\neg}$B$)\&C $
        \item ${\neg}$(${\neg} ({\neg}{\neg}$A$\&{\neg}$B$)\&C $)
      \end{enumerate}
  \end{itemize}
\end{frame}

\end{document}